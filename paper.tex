\documentclass[sigconf]{acmart}

\usepackage{booktabs} % For formal tables

\newcommand\norm[1]{\lVert#1\rVert}

% Copyright
\setcopyright{rightsretained}

% DOI
\acmDOI{11.314/176_8}

% ISBN
\acmISBN{163-2011-11-694/08/94}

\acmPrice{16.00}

%Conference
\acmConference[PASSAU'17]{Hauptseminar Media Computer Science}{August 2017}{Passau, Germany} 
\acmYear{2017}
\copyrightyear{2017}

\begin{document}
\title{A Survey on Perceptual Hashing for Nearest Neighbor Search}
\subtitle{Hauptseminar Media Computer Science, University of Passau}

\author{Mathieu Gaillard}
\orcid{1234-5678-9012}
\affiliation{%
  \institution{INSA Lyon - University of Passau}
}
\email{mathieu.gaillard@insa-lyon.fr}

\begin{abstract}
Similarity search is widely used in Information Retrieval. Its goal is to search in a large database all the documents whose distances to a query are the smallest. Many approaches allow for similarity searches that are faster than sequentially comparing all documents to a query. One of these approaches is based on hashing the documents into binary codes because they facilitate search. In this paper, we review several methods classified in two categories: data independent approaches, which make no prior assumption about the data distribution, and machine learning approaches, which take advantage of the data distribution to optimize the hash function. We compare the methods in term of hash functions, partitioning of the input space and retrieval performance.
\end{abstract}

%
% The code below should be generated by the tool at
% http://dl.acm.org/ccs.cfm
% Please copy and paste the code instead of the example below. 
%
\begin{CCSXML}
<ccs2012>
	<concept>
		<concept_id>10010147.10010257.10010258.10010260.10010271</concept_id>
		<concept_desc>Computing methodologies~Dimensionality reduction and manifold learning</concept_desc>
		<concept_significance>500</concept_significance>
	</concept>
	<concept>
		<concept_id>10010147.10010257.10010293.10010294</concept_id>
		<concept_desc>Computing methodologies~Neural networks</concept_desc>
		<concept_significance>300</concept_significance>
	</concept>
</ccs2012>
\end{CCSXML}

\ccsdesc[500]{Computing methodologies~Dimensionality reduction and manifold learning}
\ccsdesc[300]{Computing methodologies~Neural networks}

\keywords{Dimensionality reduction, Hashing, Binary codes, Machine Learning, Similarity Search}

\maketitle

\input{paper-body}

\bibliographystyle{ACM-Reference-Format}
\bibliography{paper-bibliography} 

\end{document}
